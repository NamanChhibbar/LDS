\section{Related Work}
\label{sec:related-work}

In this section, we discuss some related works that give insights into the problem.


\subsection*{Question-Answering on Long Videos}
\label{sub:long-videos}

\citet{wang2024videoagent} introduce the VideoAgent, an AI agent designed to answer a
given question using the context of a long video (approximately an hour long).
The agent first generates captions from multiple uniformly sampled frames from the
video, which are used to answer the question.
If the agent feels that the captions can not answer the question, it identifies a
segment between the initial frames to obtain more frames.

This work is relevant since a long video, a sequence of frames, can be considered
analogous to a long document, a sequence of tokens. The techniques utilized by
VideoAgent can be adapted to perform query-based summarization in long documents.


\subsection*{Long Chinese news classification}
\label{sub:chinese}

\citet{chen2022long} describe a novel algorithm for long document classification
aimed to classify Chinese news into a set of pre-defined categories.

Their approach begins with pre-processing a long text by segmenting it into sentences
and forming groups of sentences determined by a fixed maximum number of tokens a
group should have.
BERT is then used to encode these groups of sentences for further processing.
Each sentence embedding is passed through a 1D convolution layer for local feature
extraction, followed by a 1D max pooling layer.
Finally, a classifier head with softmax activation is used to classify using the
extracted features.

Methods utilized in this video, like segmenting and convolving, may prove helpful in
encoding long documents.
