\section{Future Work}
	\label{sec:future-work}

	We have used a very basic sentence tokenizer
	(\href{https://www.nltk.org/api/nltk.tokenize.sent_tokenize.html}{nltk.sent\_tokenize})
	with some modifications to control the minimum number of words in a segment, to segment
	the document.
	In our experiments, we find that segmentation is a crucial step in the pipeline and can
	influence the output summary significantly.
	Ensuring the uniformity of the length of the segments while preserving coherence within a
	segment is important for better utilization of the context size of the model while sampling.
	We encourage future work to experiment with more complex segmenters.

	Future work can also be focused on extending the Summarization with Keyword Extraction
	(\ref{method:keyword}) method.
	Experimenting with different ways to use the keywords and extraction algorithms can also be
	beneficial.


\section{Conclusion}
\label{sec:conclusion}

	Our experiments show that Document Skimming with post-sampling removal (\ref{method:skimming})
	performs well while being efficient.
	The Central Truncation method (\ref{method:truncation}) also shows good results, which
	shows that simple methods can also be effective in long document summarization.
	The last two methods, Skimming with pre-sampling removal and Summarization with Keyword
	Extraction (\ref{method:keyword}), achieve the best results but are computationally expensive.

	Our experiments show a huge jump in BERTScore compared to Unlimiformer on documents with
	word counts up to 70,000.
	This shows that our pipelines are able to utilize details in long texts efficiently.
	Even though our ROUGE-2 scores are lower than the baselines, ROUGE-1 and ROUGE-L scores are
	competitive.
	Since BERTScore is better at capturing semantic similarity, we claim that BERTScore is a
	better metric for evaluating summarizies than ROUGE scores.
	Hence, we hypothesize that our pipelines can generate better summaries than the baselines
	with higher ROUGE scores.
	It should also be noted that the models used in our experiments have smaller context sizes
	compared to the baselines, indicating that our algorithms have a greater potential if
	used with larger models.


\section*{Acknowledgement}

	All work herein reported is supported by the Nation Science Foundation under Grant
	No. 2349452.

	Any opinion, finding, or conclusion in this study is that of the authors and does not
	necessarily reflect the views of the National Science Foundation.
